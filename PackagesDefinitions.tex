%###############################################################################
% Use packages part :
%###############################################################################
\usepackage[utf8]{inputenc}
\usepackage[french]{babel}
\usepackage{graphicx}
\usepackage[T1]{fontenc}
\usepackage{amsmath, amssymb}
\usepackage{amsfonts}
%\usepackage[left=2.5cm,right=2.5cm,top=2cm,bottom=2cm]{geometry}
\usepackage{cases}
\usepackage{hyperref} % to do references
\usepackage{url} % to include urls in the text
\usepackage{xcolor}
\usepackage{subcaption} % to include subcaptions to figures
\usepackage{todonotes} % Pour \todo{toto} et \missingfigure{toto}
\usepackage{mathtools}
\usepackage{multicol} % To write in two columns
\usepackage{fancyhdr} %custom footers and headers
\usepackage{sectsty} % custom section colors
\usepackage{pdfpages} % To insert pdf files
\usepackage{tcolorbox}	% Pour écrire dans des boîtes comme en beamer
\usepackage{float} % Pour forcer la position des images

% Nomenclature
\usepackage[french]{nomencl}
\makenomenclature

%temp
\usepackage{titlesec}

% To include R code
\usepackage{listings}
\usepackage{color}
\lstset{ %
  language=R,                     % the language of the code
  basicstyle=\footnotesize,       % the size of the fonts that are used for the code
  numbers=left,                   % where to put the line-numbers
  numberstyle=\tiny\color{gray},  % the style that is used for the line-numbers
  stepnumber=1,                   % the step between two line-numbers. If it's 1, each line
                                  % will be numbered
  numbersep=5pt,                  % how far the line-numbers are from the code
  backgroundcolor=\color{white},  % choose the background color. You must add \usepackage{color}
  showspaces=false,               % show spaces adding particular underscores
  showstringspaces=false,         % underline spaces within strings
  showtabs=false,                 % show tabs within strings adding particular underscores
  frame=single,                   % adds a frame around the code
  rulecolor=\color{black},        % if not set, the frame-color may be changed on line-breaks within not-black text (e.g. commens (green here))
  tabsize=2,                      % sets default tabsize to 2 spaces
  captionpos=b,                   % sets the caption-position to bottom
  breaklines=true,                % sets automatic line breaking
  breakatwhitespace=false,        % sets if automatic breaks should only happen at whitespace
  title=\lstname,                 % show the filename of files included with \lstinputlisting;
                                  % also try caption instead of title
  keywordstyle=\color{blue},      % keyword style
  commentstyle=\color{dkgreen},   % comment style
  stringstyle=\color{mauve},      % string literal style
  escapeinside={\%*}{*)},         % if you want to add a comment within your code
  morekeywords={*,...}            % if you want to add more keywords to the set
}

%###############################################################################
% Définition de couleurs
%###############################################################################

%% couleurs berkeley
\xdefinecolor{dblue}{named}{DodgerBlue}
\definecolor{BerkeleyGold}{RGB}{193,129,45}
\definecolor{BerkeleyBlueText}{RGB}{48,99,126}
\definecolor{BerkeleyTableGold}{RGB}{222,157,46}
\definecolor{BerkeleyTableBlue1}{RGB}{207,226,234}
\definecolor{BerkeleyTableBlue2}{RGB}{232,241,245}
\definecolor{dkgreen}{rgb}{0,0.6,0}
\definecolor{gray}{rgb}{0.5,0.5,0.5}
\definecolor{mauve}{rgb}{0.58,0,0.82}

%% couleur adobe palette theme 11
\definecolor{Orange11}{RGB}{242,120,75}
\definecolor{Marron11}{RGB}{166,82,51}
\definecolor{Beige11}{RGB}{242,220,201}
\definecolor{BleuClair11}{RGB}{56,146,166}
\definecolor{BleuClair11}{RGB}{5,84,115}

%% couleurs stylées
\definecolor{Orange_FlatUI}{RGB}{231,76,60}
\definecolor{Blue_FlatUI}{RGB}{44,62,80}

\allsectionsfont{\color{Blue_FlatUI}}



\addto\captionsfrench{\def\tablename{Tableau}} % Pour renommer Table en Tableau
\usepackage{bm} % Gras dans les équations


%###############################################################################
% Maths shortcuts and definitions
%###############################################################################

\makeatletter
\DeclareRobustCommand{\pder}[1]{%
  \@ifnextchar\bgroup{\@pder{#1}}{\@pder{}{#1}}}
\newcommand{\@pder}[2]{\frac{\partial#1}{\partial#2}}
\makeatother


% Pour les valeurs absolues et normes
\newcommand{\norme}[1]{\left\Vert #1\right\Vert}

\DeclarePairedDelimiter\abs{\lvert}{\rvert}%
\makeatletter
\let\oldabs\abs
\def\abs{\@ifstar{\oldabs}{\oldabs*}}
\makeatother
